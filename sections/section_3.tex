\section{Από που προέρχεται;}
    \paragraph{}
    Αντίθετα με αυτό που θεωρεί η πλειοψηφία, το Lorem Ipsum δεν είναι απλά ένα τυχαίο κείμενο. Οι ρίζες του
    βρίσκονται σε ένα κείμενο Λατινικής λογοτεχνίας του 45 π.Χ., φτάνοντας την ηλικία του πάνω από 2000 έτη.
    Ο Richard McClintock, καθηγητής Λατινικών στο κολλέγιο Hampden-Dydney στην Βιρτζίνια, αναζήτησε μία από
    τις πιο σπάνιες Λατινικές λέξεις, την consectetur, από ένα απόσπασμα του Lorem Ipsum, και ανάμεσα σε όλα
    τα έργα της κλασσικής λογοτεχνίας, ανακάλυψε την αναμφισβήτητη πηγή του. To Lorem Ipsum προέρχεται από
    τις ενότητες 1.10.32 και 1.10.33 του "de Finibus Bonorum et Malorum" (Τα άκρα του καλού και του κακού) από
    τον Cicero (Σισερό), γραμμένο το 45 π.Χ. Αυτό το βιβλίο είναι μία διατριβή στην θεωρία της Ηθικής, πολύ
    δημοφιλής κατά την αναγέννηση. Η πρώτη γραμμή του Lorem Ipsum, "Lorem ipsum dolor sit amet...", προέρχεται
    από μία γραμμή στην ενότητα 1.10.32.

    \begin{listing}[ht]
    \caption{Απόσπασμα κώδικα σε γλώσσα Python}
    \label{listing:python example}

    \begin{minted}{python}
    import numpy as np
        
    def incmatrix(genl1,genl2):
        m = len(genl1)
        n = len(genl2)
        M = None #to become the incidence matrix
        VT = np.zeros((n*m,1), int)  #dummy variable
        
        #compute the bitwise xor matrix
        M1 = bitxormatrix(genl1)
        M2 = np.triu(bitxormatrix(genl2),1) 
        
        for i in range(m-1):
            for j in range(i+1, m):
                [r,c] = np.where(M2 == M1[i,j])
                for k in range(len(r)):
                    VT[(i)*n + r[k]] = 1;
                    VT[(i)*n + c[k]] = 1;
                    VT[(j)*n + r[k]] = 1;
                    VT[(j)*n + c[k]] = 1;
        
                    if M is None:
                        M = np.copy(VT)
                    else:
                        M = np.concatenate((M, VT), 1)
        
                    VT = np.zeros((n*m,1), int)
        
        return M
    \end{minted}
    \end{listing}

% It creates a nice aesthetic to start a new chapter/section in a new page
\newpage
